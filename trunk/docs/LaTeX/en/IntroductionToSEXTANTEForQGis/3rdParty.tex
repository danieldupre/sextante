\chapter{Configuring external applications}

\section{Introduction}

SEXTANTE can be extended using additional applications, calling them from within SEXTANTE. Currently, SAGA, GRASS and R are supported. This chapter will show you how to do it. Once you have configured the system, you will be able to execute external algorithms from any SEXTANTE component like the toolbox or the graphical modeler, just like you do with any other SEXTANTE geoalgorithm.

\subsection{A note on file formats}

When using an external software, opening a file in QGIS does not mean that it can be opened and process as well on that other woftware. In most cases, it can read what you have opened in QGIS, but in some cases, that might not be the case. When using databases or uncommon file formats, whether for raster of vector layers, problems might arise. If that happens, try to use well known file formats that you are sure that are understood by both programs, and check to console output (in the history and log dialog) for knowing more about what is going wrong.

Currently, if the layer you want to process comes from a remote service or a database connection, it cannot be processed by algorithms coming from SGA, GRASS or R, but we are working on solving this issue...(stay tuned for the next version, which might probably be able to do it)

Regarding output formats, raster layers can be saved as TIFF (.tif) (the default format) or ESRI ASCII files (.asc), while vector layers are saved as shapefiles (.shp). These have been chosen as the \emph{lingua franca} between supported third party applications and QGIS. Providers not using external application can process anylayer that you can open in QGIS, since they open it for analysis trough QGIS.


\section{SAGA}

SAGA algorithms can be run from SEXTANTE if you have SAGA installed in your system and you configure SEXTANTE properly so it can find SAGA executables. In particular, the SAGA command--line executable is needed to run SAGA algorithms. SAGA binaries are not included with SEXTANTE, so you have to download and install the software yourself. Please check the SAGA website at \texttt{} for more information.

Once SAGA is installed, open the SEXTANTE configuration dialog. In the \emph{SAGA} block you will find a setting named \emph{SAGA Folder}. Enter the path to the folder when saga is installed. Close the configuration dialog and now you are ready to run SAGA algorithms from SEXTANTE.

Notice that, ever before doing that, SAGA algorithms are shown in the toolbox and you can open them to fill the corresponding parameters dialog. However, if you try to run the algorithm after entering the parameter values, SEXTANTE will show an error message. This is because the algorithm descriptions (needed to create the parameters dialog and give SEXTANTE the information it needs about the algorithm) are not included with SAGA, but with SEXTANTE instead. That is, they are part of SEXTANTE, so you have them in yur installation even if you have not installed SEXTANTE. Running the algorithm, however, needs SAGA binaries installed in your system.

\subsection{About SAGA grid system limitations}

Most of SAGA algorithms that require several input raster layers, require them to have the same grid system. That is, to cover the same geographic area and have the same cellsize, so their corresponding grids match. When calling SAGA algorithms from SEXTANTE, you can use any layer, regarless of its cellsize and extent. When multiple raster layers are used as input for a SAGA algorithm, SEXTANTE resamples them to a common grid system and then passes them to SAGA.

The definition of that common grid system is controlled by the user, and you will find several parameters in the SAGA group of the setting window to do so. There are two ways of setting the target grid system:

\begin{itemize}
	\item{Setting it manually}. You define the extent setting the values of the following parameters:
	\begin{itemize}
		\item Resampling min X
		\item Resampling max X
		\item Resampling min Y
		\item Resampling max Y
		\item Resampling cellsize
	\end{itemize}
	Notice that SEXTANTE will resample input layers to that extent, even if they do not overlap with it.
	\item Setting it automatically from input layers. To select this option, just check the ''Use min covering grid system for resampling'' option. All the other settings will be ignored and the minimum extent that covers all the input layers will be used. The cellsize of the target layer is the maximum of all cellsizes of the input layers.
\end{itemize}

For algorithms that do not use multiple raster layers, or for those that do not need a unique input grid system, no resampling is performed before calling SAGA, and those parameters are not used.

\subsection{About vector layer selections}

By default, when a SAGA algorithm takes a vector layer, it will use all its features, even if a selection exist in QGIS. You can make SAGA aware of that seelction by checking the \emph{Use selected features} item in the SAGA settings group. When you do so, each time you execute a SAGA algorithm that uses a vector layer, the selected features of that layer will be exported to a new layer, and SAGA will work with that new layer instead.

Notice that if you select this option, a layer with no selection will behave like a layer with all its features selected, not like an empty layer.

\section{R. Creating R scripts}\label{rscripts}

R integration in SEXTANTE is different from that of SAGA in that there is not a predefined set of algorithms you can run (except for a few examples). Instead, you should write your scripts and call R commands, much like you would do from R, and in a very similar manner to what we saw in the chapter dedicated to SEXTANTE scripts. This chapter shows you the syntax to use to call those R commands from SEXTANTE and how to use SEXTANTE objects (layers, tables) in them.

The first thing you have to do, as we saw in the case of SAGA, is to tell SEXTANTE where you R binaries are located. You can do so using the \emph{R folder} entry in the SEXTANTE configuration dialog. Once you have set that parameter, you can start creating your own R scripts and executing them.

To add a new algorithm that calls an R function (or a more complex R script that you have developed and you would like to have available from SEXTANTE), you have to create a script file that tells SEXTANTE how to perform that operation and the corresponding R commands to do so.

Script files have the extension \texttt{rsx} and creating them is pretty easy if you just have a basic knowledge of R syntax and R scripting. They should be stored in the R scripts folder. You can set this folder in the R settings group (available from the SEXTANTE settings dialog), just like you do with the folder for regular SEXTANTE scripts. 

Let's have a look at a very simple file script file, which calls the R method \texttt{spsample} to create a random grid within the boundary of the polygons in a given polygon layer. This method belong to the \texttt{maptools} package. Since almost all the algorithms that you might like to incorporate into SEXTANTE will use or generate spatial data, knowledge of spatial packages like \texttt{maptools} and, specially, \texttt{sp}, is mandatory.

\begin{verbatim}
##polyg=vector
##numpoints=number 10
##output=output vector
##sp=group
pts=spsample(polyg,numpoints,type="random")
output=SpatialPointsDataFrame(pts, as.data.frame(pts))
\end{verbatim}


The first lines, which start with a double Python comment sign (\#\#), tell SEXTANTE the inputs of the algorithm described in the file and the outputs that it will generate. They work exactly with the same syntax as the SEXTANTE scripts that we have already seen, so they will not be described here again. Check the corresponding chapter for more information.

When you declare an input parameter, SEXTANTE uses that information for two things: creating the user interface to ask the user for the value of that parameter and creating a corresponding R variable that can be later used as input for R commands

In the above example, we are declaring an input of type \texttt{vector polygon} named \texttt{polyg}. When executing the algorithm, SEXTANTE will open in R the layer selected by the user and store it in a variable also named \texttt{polyg}. So the name of a parameter is also the name of the variable that we can use in R for accesing the value of that parameter (thus, you should avoid using reserved R words as parameter names).

Spatial elements such as vector and raster layers are read using the \texttt{readOGR()} and \texttt{readGDAL()} commands (you do not have to worry about adding those commands to your description file, SEXTANTE will do it) and stored as \texttt{Spatial*DataFrame} objects. Table fields are stored as strings containing the name of the selected field.

Knowing that, we can now understand the first line of our example script (the first line not starting with a Python comment).


\begin{verbatim}
pts=spsample(polyg,numpoints,type="random")
\end{verbatim}

The variable \texttt{polygon} already contains a \texttt{SpatialPolygonsDataFrame} object, so it can be usedto call the \texttt{spsample} method, just like the \texttt{numpoints} one, which indicates the number of points to add to the created sample grid.

Since we have declared an output of type vector named \texttt{out}, we have to create a variable named \texttt{out} and store a \texttt{Spatial*DataFrame} object in it (in this case, a \texttt{SpatialPointsDataFrame}). You can use any name for your intermediate variables. Just make sure that the variable storing your final result has the same name that you used to declare it, and contains a suitable value.

In this case, the result obtained from the \texttt{spsample} method has to be converted explicitly into a \texttt{SpatialPointsDataFrame} object, since it is itself an object of class \texttt{ppp}, which is not a suitable class to be retuned to SEXTANTE.

If you algorithm does not generate any layer, but a text result in the console instead, you have to tell SEXTANTE that you want the console to be shown once the execution is finished. To do so, just start the command lines that produce the results you want to print with the ``$>$'' sign. The output of all other lines will not be shown. For instance, here is the description file of an algorithms that performs a normality test on a given field (column) of the attributes of a vector layer:

\begin{verbatim}
##layer=vector
##field=field layer
##nortest=group
library(nortest)
>lillie.test(layer[[field]]) 	
\end{verbatim}

The output ot the last line is printed, but the output of the first is not (and neither are the otputs from other command lines added automatically by SEXTANTE).

If your algorithm creates any kind of graphics (using the \texttt{plot()} method), add the following line:

\begin{verbatim}
##showplots
\end{verbatim}

This will cause SEXTANTE to redirect all R graphical outputs to a temporary file, which will be later opened once R execution has finished

Both graphics and console results will be shown in the SEXTANTE results manager.

For more information, please check the script files provided with SEXTANTE. Most of them are rather simple and will greatly help you understand how to create your own ones.

A note about libraries: \texttt{rgdal} and \texttt{maptools} libraries are loaded by default so you do not have to add the corresponding \emph{library()} commands. However, other additional libraries that you might need have to be explicitly loaded. Just add the necessary commands at the beginning of your script. You also have to make sure that the corresponding packages are installed in the R distribution used by SEXTANTE.










